\documentclass[a4paper]{article}
\usepackage[utf8]{inputenc}
\usepackage[ngermanb]{babel}
\usepackage[pdfborder={0 0 0}]{hyperref}
\usepackage{listings}
\title{Famlilienmodell -- Kurzbeschreibung}
\author{Ralf Hackner und Christoph Borchert}
\date{\today}
\begin{document}

\maketitle
\tableofcontents

\section{Familien-Modell}
\subsection{Transformation des Familienmodells}
Die Transformation des Familienmodells erfolgt durch das Skript transform.pl
Dieses erwartet folgende Aufrufparameter:
\begin{description}
	\item[\texttt{-i}] (input dir) Quell-Wurzelverzeichnis
	\item[\texttt{-o}] (output dir) Ziel-Wurzelverzeichnis
	\item[\texttt{-a}] Zielarchitektur z.B. \texttt{\_tc}, \texttt{\_avr}
	\item[\texttt{-f}] Zu nutzende Feature-Auswahl (durch kconf erzeugte *.conf Datei) 
\end{description}
\subsection{Datenstrukturen}

Das Familien-Modell stellt eine Baumstruktur von Komponenten dar, deren Blätter i.d.R. die zu 
verwaltenden Dateien sind. Die Komponenten-Hierarchie kann dabei eine beliebige
Schachtelungstiefe annehmen. Die gesamte Struktur wird auf assoziative Perl-Hashes
abgebildet, wobei jede Komponente durch eine eigene Hash-Map repräsentiert wird.
Die Attribute einer Komponente werden dabei als Schlüssel-Wert-Paare dargestellt.
Kind-Komponenten werden durch Referenzen auf die entsprechende Komponente/Hash-Map in einer Liste eingefügt.
Eine Referenz auf diese wird dem Schlüssel \texttt{comp} (component) zugeordnet.\\
Ein simples Familienmodell als Beispiel:
\lstinputlisting[language=Perl,numbers=left,]{scripts/family_simple.pl}


\subsection{Elementare Attribute}
Neben dem bereits oben beschriebenen Attribut \texttt{comp}, gibt es eine Reihe
grundlegender Komponenten-Attribute:
\begin{description}
	\item[\texttt{name}] Eindeutiger Name der Komponente. Da im gesamten 
	System nur ein globaler Namensraum vorgesehen ist, ist darauf zu achten,
	dass auch bei der Verwendung mehrerer Modelle
	keine Namenskollisionen entstehen. Namenlose Komponenten sind zulässig. Es wird jedoch
	empfohlen nur Blattknoten unbenannt zu lassen, sofern diese z.B das Attribut \texttt{file}
	enthalten.
	\item[\texttt{vname}] (\textbf{v}isible \textbf{name}) Ausführlicher Name 
	der Komponente. Dient vor allem der Übersicht. Spielt keine technische
	Rolle. Dieser Name muss nicht notwendigerweise eindeutig sein.
	\item[\texttt{classname}] Klassenname ggF. mit  entsprechendem namespace
	\item[\texttt{file}] Name einer der Komponente zugeordneten Datei. Der Pfad zur
	Datei ist dabei nicht Bestandteil des Dateinamens (vgl. Pfadmechanismus). Sofern keine weiteren 
	Attribute spezifiziert sind, die ein anderes Verhalten notwendig machen,
	wird bei einer Transformation die Standard-Aktion auf die Datei angewendet.
	In der Regel wird also eine Datei dieses Namens aus dem Quellverzeichnis
	an den Zielort kopiert. Das Auffinden und Ablegen der Datei im Verzeichnisbaum
	erfolgt durch einen gesonderten Mechanismus.
	\item[\texttt{srcfile}] Sollte der Name der Quelldatei von dem der Zieldatei
	abweichen, kann mit \texttt{srcfile} ein abweichender Dateiname angegeben 
	werden.
	\item[\texttt{files}] Analog zu \texttt{file},  jedoch wird statt eines
	 einzelnen Namens eine Referenz auf eine Liste von Namen angegeben.
	 Alternativ kann auch ein regulärer Ausdruck(in Perl-Syntax) als String
	 angegeben werden (analog  zu grep -P). Berücksichtigt werden dabei alle
	 Dateien die sich am Quellpfad der Komponente oder im korrespondierenden 
	 architekturspezifischen Verzeichnis befinden.  
	 Nicht mit \texttt{srcfile} kombinierbar.
	\lstinputlisting[language=Perl]{scripts/family_files.pl}
	oder
	\lstinputlisting[language=Perl]{scripts/family_files_regexp.pl}
	
	
\end{description}

\subsection{Der Pfad-Mechanismus}
Die Pfade der Dateien werden grundsätzlich als relativ zu einem zu spezifizierenden Wurzelverzeichnis betrachtet. Im Falle von Kopiervorgängen werden ein separates Quell- und Zielwurzelverzeichnis betrachtet. Die Pfade innerhalb dieser beiden
Verzeichnisse werden als identisch betrachtet, sofern nichts anderes explizit angegeben ist. Sofern eine Komponente nicht selber 
Modifikationen am Pfad vornimmt, erbt  sie den Pfad der ihrer Eltern-Komponente. Mit folgenden Attributen kann der Pfad einer Komponente beeinflusst werden:
\begin{description}
	\item[\texttt{dir}] Wechselt in das angegebene 
	Verzeichnis relativ zum dem Skript übergebenen  Wurzelverzeichnis.
	Das Wurzelverzeichnis selbst kann 
	dabei sowohl als \texttt{'/'} oder \texttt{''} angegeben werden.
 	Abschließende Pfadseparatoren sind optional. 
	\item[\texttt{subdir}] Wechselt in das angegebene Verzeichnis relativ zum Pfad der Elternkomponente
	\item[\texttt{srcdir}] Quellpfad der Komponente bezüglich des Quellwurzelverzeichnisses, sofern dieser vom relativen Zielpfad abweicht.
\end{description}

\subsection{Von Features abhängige Komponenten}
Über das Attribut \texttt{depends} kann die Auswahl von Komponenten
eingeschränkt werden. Als Wert kann hier eine beliebige Perl
-Befehlszeile (ohne abschließenden Strichpunkt) angegeben werden.
Wird dieser zu \texttt{0} evaluiert, werden die 
Komponente und ihre Kinder nicht ausgewählt. 
Um Querverweise auf das Feature-Modell zu ermöglichen,
werden über den Autoload-Mechanismus von Perl parameterlose Funktionen mit den
Namen der Feature bereitgestellt. Diese liefern als Rückgabewert, abhängig davon
ob das Feature ausgewählt ist, \texttt{1} oder \texttt{0} zurück.
Da die Funktionen nicht explizit deklariert wurden, müssen sie entsprechend der
Perl-Syntax z.B. durch ein \texttt{\&} kenntlich gemacht werden.

\lstinputlisting[language=Perl]{scripts/family_depends.pl}

\subsection{Generatorskripten}
Mittels des Attributes \texttt{generate} können 
Dateien beliebigen Inhaltes erzeugt werden. Hierzu wird ein  Perl-Ausdruck als Wert angegeben.
Dieser wird ausgewertet, und der Rückgabewert in die Datei geschrieben.
Alle eigenen Funktionen die dazu benötigt werden, sollten in der Datei \texttt{generators.pl}
abgelegt werden.


\end{document}
